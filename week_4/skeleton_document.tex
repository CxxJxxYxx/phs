\documentclass{article}
\usepackage[margin=1in]{geometry}
\usepackage{amsmath}

\title{A very impressive title}
\author{Ann Author}


\begin{document}
\maketitle

This is a sentence. This is some math:
% enter and exit math mode with '$'. Math mode makes pretty math, non-math mode
% makes pretty text. Not being in math mode is the most common cause of LaTeX
% errors
$x^2 = \int 2x \ \text{d}x$,
also 
$e^x = \sum_{k=0}^\infty \frac{x^k}{k!}$.
Notice that even though the text
isn't on the same line in the \texttt{.tex} file, it shows up on the same line
in the pdf. However, a paragraph break is forced with an empty line betwee
paragraphs, as shown here:

I can write an entire pargraph the same as in any other txt edtior, and it will
show up as a praagraph in the finished pdf the same as in Word. In ths
paragraph I mak some little speling errors. See these errors with the command
':set spell'. Turn off spell check with ':set nospell'.  You should go throu
the paragraph and fix them using the 'f' key. In normal mode, press 'f' followd
by a letter to skip to the next use of tht letter on your current line, and
press semicolon (';') to cycle to the nxt instance. To go bcakwards, use 'F'
and semicolon (';') again. Wth 'f' you should feel the power of vim. Use it
often. We can force a paragraph break with the following command:
\\ % force a paragraph break with \\

There are many things you can modify in LaTeX including \textbf{bold} or
\textit{italic} text options, a very natural use of math symbols like $\alpha$,
$\beta$, and $\gamma$, the type of documentclass (article, letter, book,
presentation, resume, etc.), usepackages (to import special symbols or
environments that you need). Some example environments are lists
(itemize/enumerate), equations, and tables
(with tabular). Let's see one of each: \\

List with \texttt{itemize}:
\begin{itemize}
    % in itemize, '\item' makes a new item for your list
    % you can use \item[$\diamond$] to make \diamond your bullet point
    \item this is an item
    \item[$\diamond$] this is also an item
\end{itemize}

List with \texttt{enumerate}:
\begin{enumerate}

    \item now this list is enumerating
    \item and this is option two
        \begin{enumerate}
            \item and you can have lists within lists as well.
        \end{enumerate}
\end{enumerate}

If you have some set of equations that you want to include in your writing,
such as Pythagorean's theorem, which gives the hypotenuse length $c$ in terms
of the side lengths $a$ and $b$ as
\begin{equation}
    c^2 = a^2 + b^2,
\label{pythag} \end{equation}
I can now refer back to Equation \ref{pythag} whenever I want. Finally, let's
write an example table with tabular:
\begin{center}
\begin{tabular}{r|l l} % this means 3 columns, 1 right (r) justified and 2 (l) left
                       % center justified (c) is also an option
              & header 1 & header 2 \\ \hline
    subject 1 & it's good  & it's bad \\
    subject 2 & it's great & it's terrible \\
    subject 3 & it's fine  & it's okay
\end{tabular}
\end{center}

For how to include formal figures and tables, see the \texttt{example\_lab\_report.tex} file in
the \texttt{example\_tex\_report} directory.

\end{document}
